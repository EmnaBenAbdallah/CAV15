

\pref{def:PH} introduces the Process Hitting~(PH)~\cite{PMR10-TCSB}
which allows to model
%A Process Hitting (PH) (\pref{def:PH}) gathers 
a finite number of local levels,
called \emph{processes},
grouped into a finite set of components, called \emph{sorts}.
A process is noted $a_i$, where $a$ is the sort's name,
and $i$ is the process identifier within sort $a$.
At any time, exactly one process of each sort is \emph{active},
and the set of active processes is called a \emph{state}.

The concurrent interactions between processes are defined by a set of \emph{actions}.
Actions describe the replacement of a process by another of the same sort
conditioned by the presence of at most one other process in the current state.
An action is denoted by $\PHfrappe{a_i}{b_j}{b_k}$, which is read as
“$a_i$ \emph{hits} $b_j$ to make it \emph{bounce} to $b_k$”,
where $a_i$, $b_j$, $b_k$ are processes of sorts $a$ and $b$,
called respectively \emph{hitter}, \emph{target} and
\emph{bounce} of the action.
We also call a \emph{self-hit} any action whose hitter and target sorts are the same,
that is, of the form: $\PHfrappe{a_i}{a_i}{a_k}$.

\begin{definition}[Process Hitting]\label{def:PH}
  A \emph{Process Hitting} is a triple $(\PHs,\PHl,\PHa)$ where:
  \begin{itemize}
    \item  $\PHs = \{a,b,\dots\}$ is the finite set of \emph{sorts};
    \item  $\PHl = \prod_{a\in\PHs} \PHl_a$ is the set of \emph{states} where
      $\PHl_a = \{a_0,\dots,a_{l_a}\}$
      is the finite set of \emph{processes} of sort $a\in\Sigma$
      and $l_a$ is a positive integer, with $a\neq b\Rightarrow \PHl_a \cap \PHl_b = \emptyset$;
    \item  $\PHa = \{ \PHfrappe{a_i}{b_j}{b_k} \in \PHl_a \times \PHl_b^2 \mid
      (a,b) \in \PHs^2 \wedge b_j\neq b_k \wedge a=b\Rightarrow a_i=b_j\}$
      is the finite set of \emph{actions}.
  \end{itemize}
\end{definition}

%\begin{definition}[Action]
%\label{def:PhAction}
%An action is noted $\PHfrappe{a_i}{b_j}{b_k}$ where $a_i$ is a process
%of sort $a$ and $b_j$, $b_k$ two processes of sort $b$. When $a_i = b_j$ , such an action is refered as a self-action and $a_i$
%is called a self-hitting process.
%\end{definition}

\begin{example*}
\pref{fig:ph} represents a PH $(\PHs,\PHl,\PHa)$ with three sorts
($\PHs = \{a, b, c\}$) and:
$\PHl_a = \{a_0, a_1\}$,
$\PHl_b = \{b_0, b_1, b_2\}$,
$\PHl_c = \{c_0, c_1\}$.
\begin{figure}[ht]
\centering
\begin{tikzpicture}%[font=\scriptsize]
%\path[use as bounding box] (0,-1) rectangle (4,4);

\TSort{(0,0)}{z}{3}{l}
\TSort{(2,4)}{b}{2}{t}
\TSort{(4,1)}{a}{2}{r}
\THit{b_0}{}{z_1}{.east}{z_2}
\THit{b_1}{}{z_0}{.north east}{z_2}
\THit{a_0}{}{b_1}{.south}{b_0}
\THit{a_1}{out=60,in=0,selfhit}{a_1}{.east}{a_0}

\path[bounce,bend right]
\TBounce{z_1}{}{z_2}{.south}
\TBounce{z_0}{bend right=50}{z_2}{.south east}
;
\path[bounce,bend left]
\TBounce{a_1}{}{a_0}{.north}
\TBounce{b_1}{}{b_0}{.south}
;

 \THit{z_0}{}{a_0}{.west}{a_1} 

\path[bounce,bend left]
\TBounce{a_0}{}{a_1}{.south}
;
\TState{a_0,b_0,z_1}
\end{tikzpicture}
\caption{\label{fig:ph} 
A PH model example with three sorts: $a$, $b$ and $z$ ($a$ is either at level 0 or 1, $b$ at either level 0 or 1 and $z$ at either level 0, 1 or 2). Boxes represent the \emph{sorts} (network components), circles represent the \emph{processes} (component levels), and the 5 \emph{actions} that model the dynamic behavior are depicted by pairs of arrows in solid and dotted lines. The grayed processes stand for the possible initial state: $\PHstate{a_1, v_0, z_1}$.
}
\end{figure}
\end{example*}
A state of the networks is a set of active processes containing a single process of each sort.
\modMF{%
The active process of a given sort $a \in \PHs$ in a state $s \in \PHl$
is noted $\PHget{s}{a}$.
For any given process $a_i$ we also note: $a_i \in s$ if and only if $\PHget{s}{a} = a_i$.
}%

\begin{definition} [Playable action]
\label{def:playableAction}
Let $PH = (\PHs,\PHl,\PHa)$ be a Process Hitting and $s \in \PHl$ a state of $PH$. We say that the action $h = \PHfrappe{a_i}{b_j}{b_k} \in \PHa$ is \emph{playable in state $s$} if and only if $a_i \in s$ and $b_j \in s$ (\ie $\PHget{s}{a} = a_i$ and $\PHget{s}{b}=b_j$).
\end{definition}
The resulting state after playing an action $h$ in a state $s$ where it is playable is denoted by $(s \play h)$, where $\PHget{(s \play h)}{b} = b_k$ and $\forall c \in \PHs, c \neq b, \PHget{(s \play h)}{c}=\PHget{s}{c}$.

\subsection*{Dynamic properties}

The study of the dynamics of biological networks was the focus of many works, explaining the diversity of network modelings and the different methods developed in order to check dynamic properties.
In this paper we focus on 2 main properties: the stable states and the reachability.
We formally define these properties in the following and explain how they could be verified in a PH network.

The dynamics of a network or its evolution is specified by its actions.
Indeed, when an action $h = \PHfrappe{a_i}{b_j}{b_k}$ is played at a state $s$,
the network evolves to a resulting next state $s'$ where exactly one active process
is modified, namely: the active process $b_j$ changes into $b_k$.

\todo{Glu pour la définition \ref{def:NextState}}

\begin{definition}[Next states]
\label{def:NextState}
Let $(\PHs,\PHl,\PHa)$ be a Process Hitting and $s \in \PHl$ be
one of its states. The set of the next possible states for $s$ are computed as follows:
\begin{center}
$next(s) = \{s[b_k /b_j ]| \exists (a_i , b_j ) \in s^2 , \exists b_k \in \PHl_b , \PHfrappe{a_i}{b_j}{b_k} \in \PHa\}$
\end{center}
\todo{Définir la notation $s[b_k /b_j ]$}
\annotMF{Est-ce que $next(s)$ est vraiment utile ? On ne s'en ressert pas après.}
\end{definition}

A sequence of successively playable actions ($h_0 \play h_1 \play h_2...$)
from an initial state $s_0$ is called a \emph{scenario in $s_0$}.
We note $\Sce(s_0)$ the set of all scenarios in $s_0$.

\todo{Glu pour la définition \ref{def:FixPoint1} et le lemme \ref{def:FixPoint2}}

\begin{definition}[Stable state or fix point]
\label{def:FixPoint1}
Let $PH = (\PHs,\PHl,\PHa)$ be a Process Hitting and
$s \in \PHl$ be state;
$s$ is a \emph{stable state} for $PH$ if and only if $next(s) = \emptyset $.
\annotMF{“Stable state” ? “Fix point” ? “Fixed point” ?}
\end{definition}

\begin{definition}
[Lemma: Stable state or fix point]
\label{def:FixPoint2}
Let $PH = (\PHs,\PHl,\PHa)$ be a Process Hitting and
$s \in \PHl$ be a state;
$s$ is a stable state for $PH$ if and only if there is no playable action in state $s$: 
\begin{center}
$\forall h \in \PHa$, $frappeur(h) \notin s \vee cible(h) \notin s$
\end{center}
\end{definition}

\todo{Glu pour la définition \ref{def:reachability}}

\begin{definition}[Reachability question]
\label{def:reachability}

 If $s \in \PHl$ is a state and $A \subseteq \Proc$ a set of processes,
 we note $\mathcal{P}(s, A)$ the \emph{the reachability question}:
 \begin{center}
  $\exists? \delta \in \Sce(s), \forall a_i \in A, \PHget{(s \play \delta)}{a}=a_i$
 \end{center}
\todo{Il faut définir $\Proc$}
\annotMF{Il serait peut-être mieux de parler de “reachability property” plutôt que “question”.}
\end{definition}


