

\pref{def:PH} introduces the Process Hitting~(PH)~\cite{pauleve2011modelisation}
which allows to model
%A Process Hitting (PH) (\pref{def:PH}) gathers 
a finite number of local levels,
called \emph{processes},
grouped into a finite set of components, called \emph{sorts}.
A process is noted $a_i$, where $a$ is the sort's name,
and $i$ is the process identifier within sort $a$.
At any time, exactly one process of each sort is \emph{active},
and the set of active processes is called a \emph{state}.

The concurrent interactions between processes are defined by a set of \emph{actions}.
Actions describe the replacement of a process by another of the same sort
conditioned by the presence of at most one other process in the current state.
An action is denoted by $\PHfrappe{a_i}{b_j}{b_k}$, which is read as
``$a_i$ \emph{hits} $b_j$ to make it bounce to $b_k$'',
where $a_i,b_j,b_k$ are processes of sorts $a$ and $b$,
called respectively \emph{hitter}, \emph{target} and
\emph{bounce} of the action.
We also call a \emph{self-hit} any action whose hitter and target sorts are the same,
that is, of the form: $\PHfrappe{a_i}{a_i}{a_k}$.

\begin{definition}[Process Hitting]\label{def:PH}
  A \emph{Process Hitting} is a triple $(\PHs,\PHl,\PHa)$:
  \begin{itemize}
    \item[--]  $\PHs = \{a,b,\dots\}$ is the finite set of \emph{sorts};
    \item[--]  $\PHl = \prod_{a\in\PHs} \PHl_a$ is the set of states with
      $\PHl_a = \{a_0,\dots,a_{l_a}\}$
      the finite set of \emph{processes} of sort $a\in\Sigma$
      and $l_a$ a positive integer, with $a\neq b\Rightarrow \PHl_a \cap \PHl_b = \emptyset$;
    \item[--]  $\PHa = \{ \PHfrappe{a_i}{b_j}{b_k} \in \PHl_a \times \PHl_b^2 \mid
      (a,b) \in \PHs^2 \wedge b_j\neq b_k \wedge a=b\Rightarrow a_i=b_j\}$
      is the finite set of \emph{actions}.
  \end{itemize}
\end{definition}

\begin{definition}[Action]
\label{def:PhAction}
An action is noted $\PHfrappe{a_i}{b_j}{b_k}$ where $a_i$ is a process
of sort $a$ and $b_j$, $b_k$ two processes of sort $b$. When $a_i = b_j$ , such an action is refered as a self-action and $a_i$
is called a self-hitting process.
\end{definition}

\begin{example*}
\pref{fig:ph} represents a PH $(\PHs,\PHl,\PHa)$ with three sorts
($\PHs = \{a, b, c\}$) and:
$\PHl_a = \{a_0, a_1\}$,
$\PHl_b = \{b_0, b_1, b_2\}$,
$\PHl_c = \{c_0, c_1\}$.
\begin{figure}[ht]
\centering
\begin{tikzpicture}%[font=\scriptsize]
%\path[use as bounding box] (0,-1) rectangle (4,4);

\TSort{(0,0)}{z}{3}{l}
\TSort{(2,4)}{b}{2}{t}
\TSort{(4,1)}{a}{2}{r}
\THit{b_0}{}{z_1}{.east}{z_2}
\THit{b_1}{}{z_0}{.north east}{z_2}
\THit{a_0}{}{b_1}{.south}{b_0}
\THit{a_1}{out=60,in=0,selfhit}{a_1}{.east}{a_0}

\path[bounce,bend right]
\TBounce{z_1}{}{z_2}{.south}
\TBounce{z_0}{bend right=50}{z_2}{.south east}
;
\path[bounce,bend left]
\TBounce{a_1}{}{a_0}{.north}
\TBounce{b_1}{}{b_0}{.south}
;

 \THit{z_0}{}{a_0}{.west}{a_1} 

\path[bounce,bend left]
\TBounce{a_0}{}{a_1}{.south}
;
\TState{a_0,b_0,z_1}
\end{tikzpicture}
\caption{\label{fig:ph} 
A PH model example with three sorts: $a$, $b$ and $z$($a$ is either at level 0 or 1, $b$ at either level 0 or 1 and $z$ at either level 0, 1 or 2).
Circles represent the processes, boxes represent the sorts, and the actions are drawn by pairs of arrows in solid and dotted lines.The grayed processes stand for a possible initial state.
}
\end{figure}

\end{example*}

\begin{definition}[Next state]
\label{def:NextState}
Let $(\PHs,\PHl,\PHa)$ be a Process Hitting and $s \in \PHl$ be
one of its states. The set of the next possible states for $s$ are computed as follows:
\begin{center}
$next(s) = {s[b_k /b_j ]| \exists (a i , b j ) \in s^2 , \exists b_k \in \PHl_b , \PHfrappe{a_i}{b_j}{b_k} \in \PHa}$
\end{center}

\end{definition}

\begin{definition}[Stable state or fixed point]
\label{def:FixPoint}
Let $PH = (\PHs,\PHl,\PHa)$ be a Process Hitting and
$s \in \PHl$ be a state, $s$ is a stable state for $PH$ if and only if $next(s) = \emptyset $.
\end{definition}

\begin{definition}[Reachability]
\label{def:Reachability}
Let $PH = (\PHs,\PHl,\PHa)$ be a Process Hitting and
$s \in \PHl$ be a state OF $PH$, $s$ is a stable state for $PH$ if and only if $next(s) = \emptyset $. (i.r there is no playable action at the state $s$).
\end{definition}

\begin{definition} [Playable action]
\label{def:playableAction}
Let $PH = (\PHs,\PHl,\PHa)$ be a Process Hitting and $s \in \PHl$ a state of $PH$. We say that the action $h = \PHfrappe{a_i}{b_j}{b_k} \in \PHa$ is playable at the state $s$ if and only if $hitter(h)=a_i ~ \in s$ and $target(h)=b_j \in s$ (i.e $s[a]=a_i$ et $s[b]=b_j$ ) \\
The resulting state after playing the action $h$ at $s$ is denoted by $(s \play h)$ or $(s \play h)[b]=b_k$ and $\forall c \in \PHs, ~ c \neq b, (s \play h)[c]=s[c].$
\end{definition}
