%Section: Dynamic network evolution

In this section, firstly present how to determine the possible evolution of a PH model after a finite number of steps with an ASP program.
Then we tackle the reachability question: are there scenarios in a given initial state
that allow to reach a given goal (under the form of a set of processes)?

\subsection{Future states identification}
\reformuler{From an initial known state, a PH network can evolve into several new states after a few steps.}
\modMF{%
Firstly, we focus on the evolutions of the model in a limited number of steps.
}%
The predicate \texttt{time(0..n)} sets the number of steps we want to play.
\modMF{%
The value of \texttt{n} can be arbitrarily chosen.
For example, \texttt{time(0..10)} will be used to compute the 11 first steps,
including the initial state.
In order to specify such an initial state, we add several facts of the following form
to list the processes included in this state:
}%
\begin{lstlisting}
init(activeProcess("a",0)).
\end{lstlisting}
where \texttt{"a"} is the name of the sort and\texttt{"0"} the index of the active process.
The dynamics of a network is described by its actions,
identifying the future states requires to first identify the playable actions for each state.
We recall that an action is playable in a state when both its hitter and target are active in this state (see Definition \ref{def:playableAction}).
Therefore, we define an ASP predicate \texttt{playableAction(A,I,B,J,K,T)} that is true
when the processes \texttt{(A,I)} and \texttt{(B,J)} are active at step \texttt{T}.
\annotMF{Peut-être écrire : $\texttt{A}_\texttt{I}$ et $\texttt{B}_\texttt{J}$}
It is also needed to enforce the strictly asynchronous dynamic
which state that exactly one process can change between two steps.
We thus represent the change of the active process of a sort
by the predicate \texttt{"activeFromTo(B,J,K,T)"}
which means that in sort \texttt{B}, the active process changes from index \texttt{J} to \texttt{K} between steps \texttt{T} and \texttt{T+1}.
\annotMF{Peut-être écrire : $\texttt{B}_\texttt{J}$ et $\texttt{B}_\texttt{K}$}
\modMF{%
The cardinality rule of line \ref{e2} then
created as many answer sets as there are possible evolutions from the current step,
thus recreating all possible branchings in the possible evolutions of the model.
This allows to filter all scenarios where two actions have been played between
two steps, by using the constraint of line \ref{e3}.
Thus, the remaining scenarios of the answer set all strictly follow
the asynchronous dynamics of the PH.
}%
\begin{lstlisting}
{activeFromTo(B,J,K,T)} :-  playableAction(A,I,B,J,K,T),
          instate(activeProcess(A,I),T), instate(activeProcess(B,J),T),
          J!=K, time(T). %\label{e2}
:- 2{ activeFromTo(B,J,K,T)}, time(T). %\label{e3}
\end{lstlisting}
\annotMF{N'est-il pas possible de réunir ces deux règles avec des bornes 1--1 de la façon suivante : 1 \{activeFromTo(B,J,K,T)\} 1 $\leftarrow$ ... ? Sinon, voir à peut-être enlever la cardinalité pour une règle plus simple à comprendre.}

Finally, the active processes at step \texttt{T+1},
that represent the next state depending on the chosen bounce,
can be computed by the following rules:
\begin{lstlisting}
instate(activeProcess(B,K),T+1) :-  activeFromTo(B,J,K,T), time(T). %\label{e4}
instate(activeProcess(A,I),T+1) :-  instate(activeProcess(A,I),T),
          activeFromTo(B,J,K,T), A!=B, time(T). %\label{e5}
\end{lstlisting}
In other words, the state of step \texttt{T+1} contains one new active process
resulting from the predicate \texttt{activeFromTo} (line \ref{e4})
as well as all the unchanged processes that correspond to the other sorts (line \ref{e5}).

\modMF{%
The overall result of the pieces of program presented in this subsection
is an answer set containing one answer for each
possible evolution in \texttt{n} time steps,
and starting from a given initial state.
}%

\subsection{Reachability verification}
In this section, we focus on the reachability of set of processes which corresponds to the question (see definition \ref{def:reachability}):
“Is it possible, starting from a given initial state, to play a number of actions so that a set of given processes are active in the resulting state?”
\annotMF{En fin de rédaction, il faudrait voir à peut-être décaler les définitions de la section “Dynamic properties” afin de faciliter la lecture.}

We now want to adapt the code of the dynamics computation of previous section in order to resolve this reachability problem.
For this, we first use a predicate to list the processes we want to reach, called \texttt{goal}, and we add as many rules of the following form as there are objective processes:
\begin{lstlisting}
goal(activeProcess("a",1)). %\label{c1}
\end{lstlisting}
\modMF{%
Then, the literal \texttt{satisfiable(F,T)}
checks if a given process \texttt{F} of the goal
is contained in the state of step \texttt{T},
as defined in the rule of line \ref{c1}.
}%
Else the answer will be eliminated.
\todo{Il manque la contrainte qui élimine les mauvais parcours !}
\begin{lstlisting}
satisfiable(F,T) :-  goal(F), instate(F,T). %\label{c2}
\end{lstlisting}

However, the limitation of the method above is that the user has to decide upstream
the number of computed steps that should be sufficient.
It is a main disadvantage because a search in $N$ steps will find no solution
if the shortest path to the goal requires $N+1$ steps.
\annotMF{Est-ce que ça ne risque pas aussi de rallonger anormalement la résolution si le plus court chemin est en $n$ étapes avec $n << N$ ?}
One solution is to use an incremental computation mode,
which is especially tackled by by the incremental solver \textsc{iclingo}~\cite{gebser2008user}.
\annotMF{Peut-on citer le manuel d'utilisation de Clingo comme référence ?}
\todo{Expliquer brièvement en quoi consiste la résolution incrémentale en ASP, notamment les trois parties distinctes : définitions, corps d'itération, fin d'itération.}
\modMF{%
Thus, the obtained program is almost identical,
except that step numbers \texttt{N}
are replaced by constant placeholders \texttt{t}
that will be incremented throughout the execution.
}%
In each step \texttt{t}, the program computes the playable actions \texttt{playableAction(A,I,B,J,K,t)}, the possible bounces \texttt{activeFromTo(B,J,K,t-1)}
and the new states \texttt{instate(activeProcess(A,I),t+1)}
in the same fashion than previously.
Regarding the part of local steps,
\todo{Définir “the part of local steps”}
we use a special constraint (line \textbf{c4})
which states that the program should continue to the next step
while it is not satisfiable,
so that all answers that do not meet the goals are eliminated:
\begin{lstlisting}
notSatisfiable(t) :-  goal(F), not instate(F,t). %\label{c3}
:- notSatisfiable(t). %\label{c4}
\end{lstlisting}
