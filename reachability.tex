%Section: Dynamic network evolution

In this section, we present at first how to determine the possible behaviour in a PH model after a finite number of steps with an ASP program.
Then we tackle the reachability question: are there scenarios from a given initial state
that allow to reach a given goal?

\subsection{Future states identification}
In the previous section, enumerating the fixed points did not require to
encode the full dynamics of PH, but only a condition.
In this section, we thus implement the dynamics of the PH into ASP,
which will then permit to search for the paths allowing to reach the goals.

Firstly, we focus on the evolutions of models in a limited number of steps.
We therefore define the predicate \texttt{time(0..n)} which sets the number of steps we want to play.
The value of \texttt{n} can be arbitrarily chosen;
for example, \texttt{time(0..10)} will be used to compute the 11 first steps,
including the initial state.
In order to specify such an initial state, we add several facts of the following form
to list the processes included in this state:
\begin{lstlisting}
init(activeProcess("a",0)).
\end{lstlisting}
where \texttt{"a"} is the name of the sort and \texttt{"0"} the index of the active process.
The dynamics of a network is described by its actions,
identifying the future states requires to first identify the playable actions for each state.
We remind that an action is playable in a state when both its hitter and target are active in this state (see Definition \ref{def:playableAction}).
Therefore, we define an ASP predicate \texttt{playableAction(A,I,B,J,K,T)} that is true
when the processes $\texttt{A}_\texttt{I}$ and $\texttt{B}_\texttt{J}$ are active at step \texttt{T}.
It is also needed to enforce the strictly asynchronous dynamic
which state that exactly one process can change between two steps.
We thus represent the change of the active process of a sort
by the predicate \texttt{activeFromTo(B,J,K,T)}
which means that in sort \texttt{B}, the active process changes from index $\texttt{B}_\texttt{J}$ to $\texttt{B}_\texttt{K}$ between steps \texttt{T} and \texttt{T+1}.
The cardinality rule of line \ref{e2a} - \ref{e2}
creates the set of the predicates as many as there are possible evolutions from the current step,
thus recreating all possible branchings in the possible evolutions of the model so many answer sets.
This allows to filter all scenarios where two actions have been played between
two steps, by using the constraint of line \ref{e3}.
Thus, the remaining scenarios of the answer sets all strictly follow
the asynchronous dynamics of the PH.
\begin{lstlisting}
{activeFromTo(B,J,K,T)} :-  playableAction(A,I,B,J,K,T), %\label{e2a}
          instate(activeProcess(A,I),T), instate(activeProcess(B,J),T),
          J!=K, time(T). %\label{e2}
:- 2 {activeFromTo(B,J,K,T)}, time(T). %\label{e3}
\end{lstlisting}

Finally, the active processes at step \texttt{T+1},
that represent the next state depending on the chosen bounce,
can be computed by the following rules:
\begin{lstlisting}
instate(activeProcess(B,K),T+1) :-  activeFromTo(B,J,K,T), time(T). %\label{e4}
instate(activeProcess(A,I),T+1) :-  instate(activeProcess(A,I),T), %\label{e5}
          activeFromTo(B,J,K,T), A!=B, time(T). %\label{e5-a}
\end{lstlisting}
In other words, the state of step \texttt{T+1} contains one new active process
resulting from the predicate \texttt{activeFromTo} (line \ref{e4})
as well as all the unchanged processes that correspond to the other sorts (line \ref{e5} and \ref{e5-a}).

The overall result of the pieces of program presented in this subsection
is an answer set containing one answer for each
possible evolution in \texttt{n} time steps,
and starting from a given initial state.

\subsection{Reachability verification}
In this section, we focus on the reachability of a set of processes which corresponds to the reachability property (see definition \ref{def:reachability}):
“Is it possible, starting from a given initial state, to play a number of actions so that a set of given processes are active in the resulting state?”
We now want to adapt the code of the dynamics computation of previous section in order to resolve this reachability problem.
For this, we first use a predicate to list the processes we want to reach, called \texttt{goal}, and we add as many rules of the following form as there are objective processes:
\begin{lstlisting}
goal(activeProcess("a",1)). %\label{c1}
\end{lstlisting}
Then, the literal \texttt{satisfiable(F, T)} 
checks if a given process \texttt{F} of the goal
is contained in the state of step \texttt{T},
as defined in the rule of line \ref{c2}.
Else the answer will be eliminated by a constraint (not detailed here) which verifies if all processes of the goal are satisfied.
\begin{lstlisting}
satisfiable(F, T) :-  goal(F), instate(F, T). %\label{c2}
\end{lstlisting}

However, the limitation of the method above is that the user has to decide upstream
the number of computed steps that should be sufficient.
It is a main disadvantage because a search in $N$ steps will find no solution
if the shortest path to the goal requires $K$ steps with $K > N$.
It may also needlessly lengthen the resolution if the shortest path requires $n$ steps with $n << N$.
One solution is then to use an incremental computation mode,
which is especially tackled by the incremental solver \textsc{iClingo}~\cite{gebser2008user}.
The syntax of \textsc{iClingo} separates the program in 3 parts.
The \texttt{\#base} part contains only non-incremental elements
and is thus used to declare general rules
that do not depend on the time step (such as the data related to the model).
The body iteration is written in the
\texttt{\#cumulative} and \texttt{\#volatile} parts
which are computed at each incremental step,
the first part comprising rules depending on the time step,
and the second containing a constraint that stops the iteration when needed.
The step number is not given by a variable but by a constant placeholder
called “\texttt{t}” in the following.

When using this new syntax, the obtained program is almost identical
to what was presented before,
except that step numbers \texttt{T}
are replaced by the constant placeholder \texttt{t}.
In each step \texttt{t}, the program computes the playable actions \texttt{playableAction(A,I,B,J,K,t)}, the possible bounces \texttt{activeFromTo(B,J,K,t-1)}
and the new states \texttt{instate(activeProcess(A,I),t+1)}
in the \texttt{\#cumulative} part
the same way than previously, but only for the current step.
The solver then compares its current answer sets with
the \texttt{t}-dependent constraint given in the \texttt{\#volatile} part.
Regarding our implementation, this constraint is given in line \ref{c4}
and simply states that all goals have to be met.
If this constraint invalidates all current answer sets,
the computation continues in the next iteration in order to reach a valid answer set.
As soon as some answer sets meet the constraints,
they are returned and the computation stops.
\begin{lstlisting}
notSatisfiable(t) :-  goal(F), not instate(F,t). %\label{c3}
:- notSatisfiable(t). %\label{c4}
\end{lstlisting}
