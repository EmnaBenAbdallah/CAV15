For more than a decade, model-checking and SAT-solvers developments have been strongly connected \cite{biere1999symbolic}. Among different reasons, the main issue is related to the different problems with BDD-based symbolic approaches, that is:
\begin{itemize}
\item BDD-based structures require some canonical form to be given so that a total order can be defined; 
\item They often become too large;
\item Variable ordering is crucial to the performances of the approaches.  
\end{itemize}

Behind the use of SAT (respectively ASP) to perform model-checking, there is the intuitive idea that the existence of a trace satisfying a given property expressed in LTL \cite{biere1999symbolic} or CTL (first reduced to universal properties \cite{penczek2002bounded}, then extended to other properties) can be reduced to the satisfiability of an equivalent logic formula (respectively a logic program). Although there have been some works in the field of unbounded model-checking \cite{mcmillan2002applying}, it generally requires to consider bounded traces, which gave birth to the concept of Bounded Model-Checking (BMC). Its principle is following: given a discrete model $M$ (which, equipped with a semantics, leads to a transition system), a property $\varphi$ and $k \in \mathbb{N}$, does $M$ allow a counterexample to $\varphi$ of $k$ (or fewer) transitions? 

Bounded model-checking has been the subject of numerous researches. On the one hand, it also suffers from some limitations. First, it leads to incomplete analysis: if the SAT solver proves that the problem given as input is unsatisfiable for a length $k$, it only proves there are no counterexamples of length $k$. Finding bounds on the length is difficult and worst case is exponential. On the other hand, SAT is a viable alternative to BDD-based symbolic model-checking. It is also an efficient approach for debugging, \ie quickly find counterexamples of minimal length. 

Recently, the rise of ASP has also given birth to some innovative approaches, taking profit of the performances of ASP to process efficient model-checking algorithms. In \cite{rocca2013inference}, the authors proposed an approach to perform LTL and CTL model-checking through ASP, with the goal to infer and learn (synchronous) Boolean networks. This method has been refined in the context of (asynchronous) Thomas' biological regulatory networks \cite{roccaasp}. 

As regulatory phenomena play a crucial role in biological systems, they need to
be studied accurately. Biological Regulatory Networks (BRNs) consist in sets
of either positive or negative mutual effects between the components. With the
purpose of analyzing these systems, they are often modeled as graphs which make
it possible to determine the possible evolutions of all the interacting components of the system. Indeed, in order to address the formal checking of dynamical properties within very large BRNs, we recently introduced a new formalism, named the “Process Hitting” (PH) \cite{PMR10-TCSB}, to model concurrent systems having components with a few qualitative levels. A PH describes, in an atomic manner, the possible evolutions of a “process” (representing one component at one level) triggered by the hit of at most one other “process” in the system. This particular structure makes the formal analysis of BRNs with hundreds of components tractable by using approached methods computing approximations of the dynamics~\cite{PMR12-MSCS}. Furthermore, PH is suitable, according to the precision of the available information, to model BRNs with different levels of abstraction by capturing the most general dynamics.

The objectives of the work presented in this paper are the following.
Firstly, we show that starting from a PH model, it is possible to find all possible
fixed points (also called stable states)
which are specific attractors with only one state~\cite{wuensche1998genomic}).
For this, we perform an exhaustive search of the possible states and then check which ones are fixed points.
The second phase of our work consists in computing the dynamics by determining, from a given initial state, the possible next states of the PH model.
This also permits to check if a goal is reachable for a given component or set of components in the system.
The results are ensured to respect strictly the PH dynamics.

Our contribution is from the results that allowed to determine the stable states, we propose to evaluate the benefits of the Answer Set Programming (ASP) \cite{baral2003knowledge} to compute them. ASP has been proven efficient to tackle models with a large number of components and parameters. Our aim here is to assess its potential w.r.t. the computation of some dynamical properties of the PH model. In this paper, we show that ASP turns out to be effective for these enumerative searches which justifies its use. The benefit of our approach is that it makes possible to get the minimal paths to reach our goal(s) also we can verify if it is possible after a given number of steps. 

Section \ref{defs} defines the paradigm of Answer Set Programming,
and the Process Hitting framework.
We then use ASP in section \ref{fixpoint} to tackle the enumeration
of all fixed points in a model,
and in section \ref{dynamics} in order to compute the dynamics of a model
and check reachability properties.
Finally, a comparison of the methods developed in this paper is
proposed in section \ref{comparison},
and in section \ref{ccl} we conclude and talk about future work.

