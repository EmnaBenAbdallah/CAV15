In this paper, we gave a new method to compute some dynamical properties
on Process Hitting models, a subclass of asynchronous automata.
The main interest of our method is the use of Answer Set Programming,
a declarative programming paradigm which benefits from powerful solvers.
We first focused on the enumeration of the fixed points of a model,
which is tackled simply on such models given their particular form.
We also considered the reachability problem, that is,
checking if it is possible to reach a state with a given property
from a given initial state,
which thus corresponds to an $\mathsf{EF}$ operator in CTL logic.
Our analysis is thus exhaustive, but can be limited to a number of steps,
for which the dynamics of the model from the given initial state is computed.
We gave an implementation of these problems into Answer Set Programming,
and applied them to several biological examples of various sizes, up to
40 biological components.
Our results showed that our implementation is faster and deals with bigger models
than other approaches, especially \textsc{LibDDD} which is a symbolic model-checker.

Our work could benefit from several extensions.
Of course, the set of applicable models can be extended,
for example with the addition
of priorities or neutralizing edges,
or by considering synchronous dynamics or other representations
such as Thomas modeling~\cite{BernotSemBRN}.
However, the range of the analysis can also be extended,
by searching instead the set of initial states
allowing to reach a given goal,
or extending the method to universal properties (like the $\mathsf{AF}$ operator).
Finally, the research of attractors in a more general fashion
(such as cyclic or complex attractors)
would be of major interest to fully understand the behavior of models.

{\small
\paragraph{Acknowledgement}
The European Research Council has provided financial support
under the European Community's Seventh Framework Programme (FP7/2007--2013)~/
ERC grant agreement no.~259267.
}
