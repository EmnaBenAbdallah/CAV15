The study of fix points (or basins of attraction) provides an important understanding of the different behaviors of a BRN (Biological Regulatory Network) \cite{wuensche1998genomic}.  
The fixed point is a stable state of the BRN in wich it is not possible any more to have new changes.
Let $(\Sigma, L, H)$ be a Process Hitting .
 It has been shown that a state $s \in L$ is a fixed point of the Process Hitting if and only if $s$ is has no next state \cite{PMR10-TCSB} i.e. there is no playable action at this state. In fact a steady state od a Process Hitting network is a set of processes with exactly one process of each sort as well as every process has no hit with the other selected ones (process with self-hit cannot be apart of a stable state).

\subsection{Process Hitting network traduction }
In order to handle a PH biological network, it was necessary first to present it with ASP. To do this we chose the predicates: \texttt{sort}, \texttt{process} and \texttt{action}. Below is an example of PH network written in ASP.

\begin{example*} [Example PH network with ASP]
If we try to present the network of Figure \ref{fig:ph} we will have:
\begin{lstlisting}
sort("a"). sort("b"). sort("z"). %\label{ASPsort}
process("a", 0..1). process("b", 0..1). process("z", 0..2). %\label{ASPprocess}
action("a",0,"b",1,0). action("a",1,"a",1,0). action("b",1,"z",0,2). %\label{actions1}
action("b",0,"z",1,2). action("z",0,"a",0,1). %\label{actions2}
\end{lstlisting}
Line \ref{ASPsort} shows every sort of network with the predicate comes out and in parentheses the name of the sort. In line \ref{ASPprocess} there is a list of processes corresponding to each \texttt{sort}, for example the sort \texttt{"z"} has 3 processes numbered from \texttt{0} to \texttt{2}, this numbering is provided by the $2^{nd}$ parameter of the predicate \texttt{process("z", 0..2)}. Finally we find all network actions that are defined in lines \ref{actions1} and \ref{actions2}, for example the first action \texttt{action("a",0,"b",1,0)} is an action from $a_0$ to $b_1$ to bound so $b$ from $b_1$ into $b_0$.
\end{example*}

\subsection{Search of fix points }
\todo{ Parler du point fixe et ses interrêts. \\
Décrire la nouvelle méthode qui permet de déterminer les états stables d'un réseau }

In fact we have to translate the definition of a stable state in a method developped in ASP. So first we eliminate all processes with self-hit and we save them in the predicate \texttt{shownProcess}:
\begin{lstlisting}
hiddenProcess(A,I) :-  action(A,I,B,J,K), A=B, process(A,I),process(B,J),
                       process(B,K). %\label{hiddenProcess}
shownProcess(A,I) :-  not hiddenProcess(A,I), process(A,I). %\label{shownProcess}
\end{lstlisting}
Then we have to browse this graph and extract all possible combinations of shown processes by choosing a process from each sort.
\begin{lstlisting}
1 { selectProcess(A,I) : showProcess(A,I) } 1 :- sort(A).
\end{lstlisting}
It now remains to check each combination of processes whether it is a fix point. For this we use a special type of ASP rules: a \textbf{constraint}. The idea of constraints is based on the fact that a solution would be eliminated if it does not satisfy the constraint. For our problem a combination is eliminated if there is an action between two of the selected process:
\begin{lstlisting}
:- hit(A,I,B,J), selectProcess(A, I), selectProcess(B, J), A!= B. %\label{contraintFix}
\end{lstlisting}

\begin{example}
Considering the last graphical presentation of a precess hitting \pref{fig:ph}, there are 3 sorts \texttt{"a"}, \texttt{"b"} have 2 levels and \texttt{"z"} has 3, so we can find $2*2*3 = 12$ states (whatever they can be reached or not). If we verify wether there exist fixed points we deduce that we have only one: $<b\_0, z\_2, a\_0>$. It is clear in \pref{fig:ph} that there is no action between each two processes. Besides our new ASP method proof this and returns also the same answer:
\begin{tabbing}
 \texttt{
 Answer 1 : fixProcess(a, 0), fixProcess(b, 0), fixProcess(z, 2)
 }
\end{tabbing}
\end{example}