The study of fixed points (and, more generally, basins of attraction) provides an important understanding of the different behaviors of a Biological Regulatory Network (BRN)~\cite{wuensche1998genomic}.
Indeed, a system will always eventually end in a basin of attraction,
and this may depend on biological switch or other complex phenomena.
A fixed point is a state of the BRN in which it is not possible any more to have new changes;
in other words, it is a basin of attraction that is composed of only one state.

In the following, we consider a Process Hitting $\PH = (\Sigma, \PHl, \PHh)$.
It has been shown that a state $s \in L$ is a fixed point (or stable state) of the Process Hitting model if and only if $s$ has no next state, \ie if there is no playable action in this state~\cite{PMR10-TCSB}.
Therefore, a stable state in a Process Hitting network is a state so that
every process does not hit or is not hit by another process in the same state.
We note that given this result, processes involved in a self-hit (an action whose hitter and target processes are the same) cannot be part of a stable state.

\subsection{Process Hitting translation in ASP}
Before analyzing the dynamics of the network,
we first need to translate the concerned PH network in ASP\footnote{All programs, including this translation and the methods describes in the following, are available online at: \url{https://github.com/EmnaBenAbdallah/verification-of-dynamical-properties_PH}}.

To do this we use the following self-describing predicates:
\texttt{"sort"} to define sorts, \texttt{"process"} for the processes and \texttt{"action"} for the network actions. We will see in example \ref{ex:asp-ph} how a PH network is defined with these predicates.

\begin{example}[Representation of a PH network in ASP]
\label{ex:asp-ph}
The representation of the PH network of figure \ref{fig:ph} in ASP is the following:
\begin{lstlisting}
process("a", 0..1). process("b", 0..1). process("z", 0..2). %\label{ASPprocess}
sort(X) :- process(X,I) %\label{ASPsort}
action("a",0,"b",1,0). action("a",1,"a",1,0). action("b",1,"z",0,2). %\label{actions1}
action("b",0,"z",1,2). action("z",0,"a",0,1). %\label{actions2}
\end{lstlisting}
In line \ref{ASPprocess} we create the list of processes corresponding to each sort,
for example the sort \texttt{"z"} has 3 processes numbered from \texttt{0} to \texttt{2};
this specific predicate will in fact expand into the three following predicates:
\texttt{process("z", 0)}, \texttt{process("z", 1)}, \texttt{process("z", 2)}.
Line \ref{ASPsort} enumerates every sort of the network from the previous information.
Finally, all the actions of the network are defined in lines \ref{actions1} and \ref{actions2};
for example, the first predicate \texttt{action("a",0,"b",1,0)} represents the action
$\PHfrappe{a_1}{b_1}{b_0}$.
\end{example}

\subsection{Search of fixed points}

The enumeration of fixed points requires to translate the definition of a stable state
in a set of ASP rules.
The first step consists of eliminating all processes involved in a self-hit;
the others are recorded by the predicate \texttt{shownProcess}:
\begin{lstlisting}
hiddenProcess(A,I) :- action(A,I,B,J,K), A=B, process(A,I),process(B,J),
                      process(B,K). %\label{hiddenProcess}
shownProcess(A,I) :- not hiddenProcess(A,I), process(A,I). %\label{shownProcess}
\end{lstlisting}
Then, we have to browse all remaining processes of this graph
in order to generate all possible states,
that is, all possible combinations of processes by choosing one process from each sort:
\begin{lstlisting}
1 { selecedtProcess(A,I) : showProcess(A,I) } 1 :- sort(A).
\end{lstlisting}
The previous line creates as many potential answer sets as there are possible states
to take into account.
Finally, the last step consists of filtering any state that is not a fixed point,
or, in other words, eliminating all candidate answer sets in which an action could be played.
For this, we use a constraint:
any solution that satisfy the body of this constraint will be removed from the answer set.
Regarding our problem, a state is eliminated if there exists an action linking two selected processes:
\begin{lstlisting}
:- hit(A,I,B,J), selectedProcess(A, I), selectedProcess(B, J), A != B. %\label{contraintFix}
\end{lstlisting}

\begin{example}[Fixed points enumeration]
The PH model of \pref{fig:ph} contains 3 sorts:
$a$ and $b$ have 2 processes and $z$ has 3; therefore, the whole model has $2*2*3 = 12$ states (whether they can be reached or not from a given initial state).
We can check that this model contains only one fixed point: $\PHstate{b_0, z_2, a_0}$.
Indeed, there is no action between each two of the processes contained is this state so no execution is possible from this state. 
In this example, no other state verifies this property.

If we execute the ASP program detailed below,
alongside with the description of the PH model given in example~\ref{ex:asp-ph},
we obtain one answer set, which matches the expected result:
\begin{lstlisting}[numbers=none]
Answer 1 : fixProcess(a, 0), fixProcess(b, 0), fixProcess(z, 2)
\end{lstlisting}
\end{example}