In this section, we show the effectiveness of our approach on some examples,
and compare it to other existing approaches.
All computations were performed on a Pentium~V, 3.2~GHz with 4~GB RAM.

\subsection{Evaluation}
To evaluate the effectiveness of our new approach,
we position ourselves with respect to existing methods dealing with different biological models.
We have chosen the following tools, that are detailed below: 
\textsc{GINsim} (Gene Interaction Network Simulation)~\cite{gonzalez2006ginsim,naldi2009logical,naldi2007decision},
\texttt{LIBDDD} (Library of Data Decision Diagrams)~\cite{thierry2009hierarchical,colange2013towards},  PINT \cite{PMR12-MSCS} and the method for CTL model-checking of Rocca's et al. \cite{roccaasp} which was developed also on ASP but for transition states networks.
Each method use a specific kind of representation\footnote{The transalation from PH from or to other models is done through the converters availbale on line: \url{https://github.com/pauleve/pint/tree/master/converters}}:
logical regulatory network for \textsc{GINsim} \footnote{\textsc{GINsim} is a computer tool for the modeling and simulation of genetic regulatory networks \url{http://ginsim.org/} },
Instantiable Transition Systems for \textsc{libDDD} is a library for symbolic model-checking of CTL \& LTL properties. \footnote{\textsc{libDDD} is \url{http://www.lip6.fr/} },
state transitions network for Rocca's et al. method
and Process Hitting (PH) for \textsc{PINT} \footnote{\textsc{PINT} is a Process Hitting tool that implements the language, simulation, formal analysis, and translation of PH models. It available online \url{http://loicpauleve.name/pint/} } as well as for our method.

For this comparative study, we focus biological network of different sizes:
a tadpole tail resorption (TTR) model with 12 biological components~\cite{khalis2009smbionet},
an ERBB receptor-regulated G1/S transition (ERBB) model with 20 components~\cite{Samaga2009}
and a T-cell receptor (TCR) signaling network of 40 components~\cite{Klamt06}.

These models were chosen to be of different sizes:
from small (12 components) to large (40 components).
We note however that the PH models considered may contain more sorts than
the original number of biological components, due to the use of
“cooperative sorts”, which allow to model Booelan gates but do not necessarily
have a biological meaning.
The different model representations required have been obtained by translations
from the PH
ensuring the conservation of the dynamical properties.
All these translations are available with \textsc{PINT} 
All results alongside with more detailed specifications of the models
are given in table \ref{tab:reachability}.
The methods and the results provided by each of them are detailed in the following.
The overall results shows that our method is efficient in computing reachability
from a given initial state;
furthermore, it sometimes provides more informations that the other existing ones.

\begin{figure}[htp]
\begin{center}
\label{tab:reachability}
\noindent%
\begin{tabular}{|l|c|c|c|c||c|c|c|c|c|}
  \hline
   Model&  \#sorts &\#procs & \#states & target & \textsc{ASP-thomas} & \textsc{Pint} & \textsc{libddd} & \textsc{GINsim} & \textsc{ASP-PH} \\
  \hline
  TTR & 12 &42 & $2^{19}$ & full state & 0m7.21s & 0m.00.97s & 0m1.151s &  0m1.001s & 0m1.90s \\
  \hline
  ERBB & 20 &152 & $2^{70}$ & full state & 0m2.448s & out &1m55.38s & 2m31.64s & 0m11.84s \\
  \hline
  ERBB & 20 &152 & $2^{70}$ & partial state & 0m2.61s & 0m0.027s &1m54.96s & - & 0m5.02s \\
  \hline
  TCR & 40 &156 & $2^{73}$ & full state & - & Inconc & out & out & 4m27.93s \\
  \hline
   TCR & 40 &156 & $2^{73}$ & partial state & - & 0m0.014s & out & - & 1m35.080s \\
  \hline
\end{tabular}
\caption{Compared performances of Rocca et al. method denoted by \textsc{ASP-thomas}, \textsc{Pint}, \textsc{libddd}, \textsc{GINsim} and our new method \textsc{ASP-PH} developped in ASP for Process Hitting networks.
For each test, this table gives the considered model,
the number of its processes and its states, the type of goal
(either a whole state or a sub-state)
and the computation time of the different methods used for the tests,
where “out” marks an execution taking too much time or memory
and “-” indicates that is not possible to do the test.
}
\end{center}
\end{figure}

\begin{itemize}

\item \textbf{ASP-THOMAS} offers the possibility to model-check CTL properties
of Thomas networks.
We note that in general, presenting such a network needs so much time because there is no converters for thomas networks to ASP. regardless of the complexity of the presentation of the graph it is clear that this approach gives a very quick result when compared with others (for the example TCR we didn't do the test). Indeed it also proves ASP is a good choice to run the graphs and check these properties.
Afterwards, this method is able to check any kind of CTL formula
(and not only the “$\mathsf{EF}$” form that we focused on in this paper) but the inconvinient thet they need to provide a number of steps greater than or equal to the number of steps after which the objectives are achieved. But this number is not always estimable.

\item \textbf{\textsc{GINsim}} is a software for the edition, simulation and analysis
of gene interaction networks.
It allows to compute all reachable stable states instantly;
however, it is not possible to compute all stable states independently from the initial state.
Regarding the reachability problem, \textsc{GINSim} only allows to check
full states, because of its approach which consists in computing
all the state-transition graph and then check if there exists a path between two given states.
Therefore, it was not possible to perform reachability checks on partial states.
Small state-transition graphs can also be displayed by this tool.

\item \textbf{\textsc{libddd}}
is a library for symbolic model-checking of CTL \& LTL properties.
It can thus especially be used to check reachability properties;
however, as opposed to our method, it does not output an execution path
solving this reachability.
In addition, it relies on the construction of the state-transition graph
which is then stored under the form of a binary decision diagram for a more efficient analysis.

This computation explains why \textsc{libddd} takes more time to respond,
and does not answer in less than 12 minutes for the biggest example
(which contains $2^{73}$ states).
Finally, \textsc{libddd} is not able to compute the stable states of a network.

\item \textbf{PINT}
is a library gathering tools and translators related to the PH.
It should be noted that Pint contains the only reachability analysis
developed so far natively for the Process Hitting,
before the method proposed in this paper.
It consists in an approximation that avoids to compute the state-transition graph;
It is thus ensured to be really efficient, which explains the fastest results,
but at the cost of possibly terminating without being conclusive (although this is rare).
However, it is not designed for goals consisting of a full state, which
may trigger an exponential process to reach an answer,
which explains the high computing times for some of them \ref{tab:reachability}
Moreover, in the case of a positive answer,
it currently does not return the execution the path achieving the desired reachability,
but only outputs its conclusion.
\end{itemize}

\subsection{Strengths and limitations of our method}

In the previous sections,
we developed new methods to check dynamical properties,
namely identifying stable states and finding all possible paths to reach a given goal.
Compared to other methods (Rocca's method, \textsc{GINsim} and \textsc{libddd}),
our method is relatively faster and also permits to study larger networks
(up to $2^{73}$ states in our tests) despite some inconclusive cases.
Indeed, our ASP program is non-conclusive in the cases where the given goal is not
reachable and the model contains loops in its dynamics
(which happens in almost every biological model).
In this case, the program will compute all infinite paths in these loops,
and never reach a goal or a fixed point.
It is still possible, however, to limit the number of iterations to an arbitrary
maximum which will be eventually reached in the case of an endless loop.
This is possible with the option \texttt{"-{}-imax=n"} of \textsc{iclingo},
where \texttt{n} is the maximum number of steps.
%%
%\begin{lstlisting}[numbers=none]
%iclingo find\_path.lp <network\_name>.lp -- imax=n
%\end{lstlisting}
%%
For example, the total number of states is an obvious maximum,
as it will never be exceeded by a minimum path,
but it is too hight to be very interesting.
The total number of sorts is a more interesting value,
under the hypothesis that each one will change its active process at most once,
which is often the case for Boolean networks;
or, with a similar reasoning, the total number of processes can be chosen.
In this case, however, a termination with no solution cannot be considered as a formal
negative answer, unless one can prove that the chosen value \texttt{n}
is bigger than the longest possible path in the state-transition graph.
Given our implementation, if the step \texttt{n} is reached,
the computation stops and there is no path to be displayed because there is no path verifies the goals after \texttt{n} changes.
