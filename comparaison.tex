In this section, we show the effectiveness of our approach on some examples. All
computations were performed on a Pentium V, 3.2 GHz with 4 GB RAM.

\subsection{Evaluation}
It should be noted that the only reachability analysis developed so far on the Process Hitting networks was implemented in the software Pint, and consists in an approximation: it is possible that it terminates but remains inconclusive (although this is rare). Moreover, it currently does not give us the path to activate the goal. But we also note that there are other methods or softwares for dynamical analysis applied to other biological network models.

We define the models used for each methos, logical regulatory network for GINsim (Gène Interaction Network
Simulation) \cite{gonzalez2006ginsim}, \cite{naldi2009logical} and \cite{naldi2007decision}, Instantiable Transition Systems for libddd (Library of Data Decision
Diagrams) \cite{thierry2009hierarchical} and \cite{colange2013towards} and state transition network for Rocca's et al. method \cite{roccaasp}.

For our comparative study we chose organic network of different sizes. As mentioned each method is applied on a specific model and translation into these models from the PH network or viceversa is done thanks to PINT \cite{pauleve2011modelisation}.
The following table prensents the comparisons run for 4 reachability analysis methods  applied to large examples:
\begin{center}

\begin{figure}
\label{tab:reachability}
\noindent
\begin{tabular}{|l||c|c|c||c|c|c|c|}
  \hline
   Model& \#procs & \#states & target & Rocca & libddd & GINsim & ASP \\
  \hline
  tetard \cite{khalis2009smbionet} & 42 & $2^{19}$ & process & 7m17.01s & XX & XX & 0m01.90s \\
  \hline
  ERBB\_G1-S \cite{Samaga2009}  & 152 & $2^{70}$ & state & - &1m55.38s & 2m01.64s & 0m11.84s \\
  \hline
  ERBB\_G1-S & 52 & $2^{70}$ & process & - &1m54.96s & - & 0m05.02s \\
  \hline
  TCRsig40 \cite{Klamt06} & 156 & $2^{73}$ & process & - & out & out & 0m05.02s \\
  \hline
\end{tabular}
\caption{Compared performances of Rocca et al. method, \textsc{libddd}, \textsc{GINsim} and our new method developped in ASP. For each test we find the name of the model, the number of its processes and its states then the target(s). The target can be a network state (set of processes) or one or some processes. "out" marks an execution asking too much time or memory and "-" marks that is not possible to do the test.
}
\end{figure}
\end{center}

This table shows the effectiveness of our method compared to existing. In fact if we start by talking about Rocca's method, actually it offers the possibility to verify all model checking properties (LTL et CTL) for transition state networks. But we note that it is very difficult to write a transition state network in ASP, for example the network of tetard \cite{khalis2009smbionet} only 42 processes, take 3 minutes to be translated from PH. 

Second the \textsc{libddd} 

Third the \textsc{GINsim} is an analysis software and simulation of genetic interaction networks. It computes the stable states instantly as welle as our ASP method do it instantly.The difference that \textsc{GINsim} does not return all stable states but only reachable ones. Regarding the reachability problem, its approach is to calculate all the transition state graph and then check if there is a path between two given states. Thus it is necessary to have the entire network transition states and both initial and final states to determine the path. It was indicated in the third line of the table \ref{tab:reachability} that GINsim could not solve the problem because it is not possible to check the reachability of just one or a few components of the network indeed must be given a final state.

%We propose to evaluate the over-approximation
%due to our way of handling hybrid dynamics. For that purpose, we consider
%the 5-dimensional linear differential

%\subsection{Benchmarks }

\subsection{Limitation}